% Options for packages loaded elsewhere
% Options for packages loaded elsewhere
\PassOptionsToPackage{unicode}{hyperref}
\PassOptionsToPackage{hyphens}{url}
\PassOptionsToPackage{dvipsnames,svgnames,x11names}{xcolor}
%
\documentclass[
  11pt,
  letterpaper,
  DIV=11,
  numbers=noendperiod,
  oneside]{scrartcl}
\usepackage{xcolor}
\usepackage[top=2cm, bottom=1.3cm, left=.5cm, right=7cm, heightrounded,
marginparwidth=5cm, marginparsep=5mm]{geometry}
\usepackage{amsmath,amssymb}
\setcounter{secnumdepth}{-\maxdimen} % remove section numbering
\usepackage{iftex}
\ifPDFTeX
  \usepackage[T1]{fontenc}
  \usepackage[utf8]{inputenc}
  \usepackage{textcomp} % provide euro and other symbols
\else % if luatex or xetex
  \usepackage{unicode-math} % this also loads fontspec
  \defaultfontfeatures{Scale=MatchLowercase}
  \defaultfontfeatures[\rmfamily]{Ligatures=TeX,Scale=1}
\fi
\usepackage[]{libertinus}
\ifPDFTeX\else
  % xetex/luatex font selection
\fi
% Use upquote if available, for straight quotes in verbatim environments
\IfFileExists{upquote.sty}{\usepackage{upquote}}{}
\IfFileExists{microtype.sty}{% use microtype if available
  \usepackage[]{microtype}
  \UseMicrotypeSet[protrusion]{basicmath} % disable protrusion for tt fonts
}{}
\makeatletter
\@ifundefined{KOMAClassName}{% if non-KOMA class
  \IfFileExists{parskip.sty}{%
    \usepackage{parskip}
  }{% else
    \setlength{\parindent}{0pt}
    \setlength{\parskip}{6pt plus 2pt minus 1pt}}
}{% if KOMA class
  \KOMAoptions{parskip=half}}
\makeatother
% Make \paragraph and \subparagraph free-standing
\makeatletter
\ifx\paragraph\undefined\else
  \let\oldparagraph\paragraph
  \renewcommand{\paragraph}{
    \@ifstar
      \xxxParagraphStar
      \xxxParagraphNoStar
  }
  \newcommand{\xxxParagraphStar}[1]{\oldparagraph*{#1}\mbox{}}
  \newcommand{\xxxParagraphNoStar}[1]{\oldparagraph{#1}\mbox{}}
\fi
\ifx\subparagraph\undefined\else
  \let\oldsubparagraph\subparagraph
  \renewcommand{\subparagraph}{
    \@ifstar
      \xxxSubParagraphStar
      \xxxSubParagraphNoStar
  }
  \newcommand{\xxxSubParagraphStar}[1]{\oldsubparagraph*{#1}\mbox{}}
  \newcommand{\xxxSubParagraphNoStar}[1]{\oldsubparagraph{#1}\mbox{}}
\fi
\makeatother


\usepackage{longtable,booktabs,array}
\usepackage{calc} % for calculating minipage widths
% Correct order of tables after \paragraph or \subparagraph
\usepackage{etoolbox}
\makeatletter
\patchcmd\longtable{\par}{\if@noskipsec\mbox{}\fi\par}{}{}
\makeatother
% Allow footnotes in longtable head/foot
\IfFileExists{footnotehyper.sty}{\usepackage{footnotehyper}}{\usepackage{footnote}}
\makesavenoteenv{longtable}
\usepackage{graphicx}
\makeatletter
\newsavebox\pandoc@box
\newcommand*\pandocbounded[1]{% scales image to fit in text height/width
  \sbox\pandoc@box{#1}%
  \Gscale@div\@tempa{\textheight}{\dimexpr\ht\pandoc@box+\dp\pandoc@box\relax}%
  \Gscale@div\@tempb{\linewidth}{\wd\pandoc@box}%
  \ifdim\@tempb\p@<\@tempa\p@\let\@tempa\@tempb\fi% select the smaller of both
  \ifdim\@tempa\p@<\p@\scalebox{\@tempa}{\usebox\pandoc@box}%
  \else\usebox{\pandoc@box}%
  \fi%
}
% Set default figure placement to htbp
\def\fps@figure{htbp}
\makeatother


% definitions for citeproc citations
\NewDocumentCommand\citeproctext{}{}
\NewDocumentCommand\citeproc{mm}{%
  \begingroup\def\citeproctext{#2}\cite{#1}\endgroup}
\makeatletter
 % allow citations to break across lines
 \let\@cite@ofmt\@firstofone
 % avoid brackets around text for \cite:
 \def\@biblabel#1{}
 \def\@cite#1#2{{#1\if@tempswa , #2\fi}}
\makeatother
\newlength{\cslhangindent}
\setlength{\cslhangindent}{1.5em}
\newlength{\csllabelwidth}
\setlength{\csllabelwidth}{3em}
\newenvironment{CSLReferences}[2] % #1 hanging-indent, #2 entry-spacing
 {\begin{list}{}{%
  \setlength{\itemindent}{0pt}
  \setlength{\leftmargin}{0pt}
  \setlength{\parsep}{0pt}
  % turn on hanging indent if param 1 is 1
  \ifodd #1
   \setlength{\leftmargin}{\cslhangindent}
   \setlength{\itemindent}{-1\cslhangindent}
  \fi
  % set entry spacing
  \setlength{\itemsep}{#2\baselineskip}}}
 {\end{list}}
\usepackage{calc}
\newcommand{\CSLBlock}[1]{\hfill\break\parbox[t]{\linewidth}{\strut\ignorespaces#1\strut}}
\newcommand{\CSLLeftMargin}[1]{\parbox[t]{\csllabelwidth}{\strut#1\strut}}
\newcommand{\CSLRightInline}[1]{\parbox[t]{\linewidth - \csllabelwidth}{\strut#1\strut}}
\newcommand{\CSLIndent}[1]{\hspace{\cslhangindent}#1}



\setlength{\emergencystretch}{3em} % prevent overfull lines

\providecommand{\tightlist}{%
  \setlength{\itemsep}{0pt}\setlength{\parskip}{0pt}}



 


\usepackage{sectsty}
\usepackage{sidenotes}
\usepackage{sidenotes}
%\renewcommand{\marginnote}[2][]{\marginpar{#2}}
\renewcommand{\marginnote}[2][]{\sidenotetext[#1]{\parbox{\marginparwidth}{#2}}}
%\renewcommand{\marginnote}[2][]{\@sidenotes@placemarginal{0}{\textsuperscript{1}~#2}}
%\renewcommand{\thempfootnote}{}
\usepackage{enumitem}
\setlist[description]{style=multiline,leftmargin=5cm,itemsep=20pt}
\sectionfont{\sectionrule{0pt}{0pt}{-4pt}{1pt}}
\subsectionfont{\sectionrule{0pt}{0pt}{-4pt}{.1pt}}
\newcommand{\ut}[2]{\underbrace{#1}_{\text{#2}}}
\newcommand{\utt}[3]{\underbrace{#1}_{\substack{\text{#2}\\\text{#3}}}}
\KOMAoption{captions}{tableheading}
\makeatletter
\@ifpackageloaded{caption}{}{\usepackage{caption}}
\AtBeginDocument{%
\ifdefined\contentsname
  \renewcommand*\contentsname{Table of contents}
\else
  \newcommand\contentsname{Table of contents}
\fi
\ifdefined\listfigurename
  \renewcommand*\listfigurename{List of Figures}
\else
  \newcommand\listfigurename{List of Figures}
\fi
\ifdefined\listtablename
  \renewcommand*\listtablename{List of Tables}
\else
  \newcommand\listtablename{List of Tables}
\fi
\ifdefined\figurename
  \renewcommand*\figurename{Figure}
\else
  \newcommand\figurename{Figure}
\fi
\ifdefined\tablename
  \renewcommand*\tablename{Table}
\else
  \newcommand\tablename{Table}
\fi
}
\@ifpackageloaded{float}{}{\usepackage{float}}
\floatstyle{ruled}
\@ifundefined{c@chapter}{\newfloat{codelisting}{h}{lop}}{\newfloat{codelisting}{h}{lop}[chapter]}
\floatname{codelisting}{Listing}
\newcommand*\listoflistings{\listof{codelisting}{List of Listings}}
\makeatother
\makeatletter
\makeatother
\makeatletter
\@ifpackageloaded{caption}{}{\usepackage{caption}}
\@ifpackageloaded{subcaption}{}{\usepackage{subcaption}}
\makeatother
\makeatletter
\@ifpackageloaded{sidenotes}{}{\usepackage{sidenotes}}
\@ifpackageloaded{marginnote}{}{\usepackage{marginnote}}
\makeatother
\usepackage{bookmark}
\IfFileExists{xurl.sty}{\usepackage{xurl}}{} % add URL line breaks if available
\urlstyle{same}
\hypersetup{
  pdftitle={Models of the Economic Impact of AI},
  pdfauthor={Tom Cunningham},
  colorlinks=true,
  linkcolor={blue},
  filecolor={Maroon},
  citecolor={Blue},
  urlcolor={Blue},
  pdfcreator={LaTeX via pandoc}}


\title{Models of the Economic Impact of AI}
\author{Tom Cunningham.}
\date{2025-11-08}
\begin{document}
\maketitle


\begin{description}
\item[This note will discuss models of AI's \emph{equilibrium} impact on
the economy.]
In the last 10 years there have been a dozen papers written trying to
predict the effects of AI that give a wide range of predictions about
the effect on productivity, wages, and inequality. I will discuss these
models and try to explain the relationship between assumptions and
predictions.\sidenote{\footnotesize The only review article I'm aware of is Lu and
  Zhou (2021) from 2021.}

We mostly doesn't discuss evidence for the \emph{size} of the
productivity effects from AI or which types of activities it will
directly impact. Instead the note focusses on the \emph{equilibrium}
effects - how society will rearrange itself to make room for AI through
changes in wages, employment, and distribution of income.
\item[Summary of predictions.]
This is a somewhat opinionated take on the models below:

\begin{enumerate}
\def\labelenumi{\arabic{enumi}.}
\tightlist
\item
  AI will increase total goods and services and, to a smaller extent,
  decrease the amount that people choose to work.
\end{enumerate}

\begin{enumerate}
\def\labelenumi{\arabic{enumi}.}
\setcounter{enumi}{1}
\tightlist
\item
  The sectors in which AI has the biggest productivity effects will
  likely come to employ fewer people, and people who have skills which
  are replaced by AI will have lower wages and employment.
\end{enumerate}

\begin{enumerate}
\def\labelenumi{\arabic{enumi}.}
\setcounter{enumi}{2}
\tightlist
\item
  AI will increase real wages but it will likely increase the returns to
  capital more (i.e.~reducing the labor share of income).
\end{enumerate}

\begin{enumerate}
\def\labelenumi{\arabic{enumi}.}
\setcounter{enumi}{3}
\tightlist
\item
  It is unclear whether AI will raise the wages of the more-educated or
  less-educated more.
\end{enumerate}

\begin{enumerate}
\def\labelenumi{\arabic{enumi}.}
\setcounter{enumi}{4}
\tightlist
\item
  If AI can outperform most humans at \emph{every} task (AGI) then it
  will cause a huge boost in output but decrease wages (the returns to
  working), because computers compete with humans for land and energy
  inputs.
\end{enumerate}
\end{description}

\marginnote{\begin{footnotesize}

\pandocbounded{\includegraphics[keepaspectratio]{2024-08-19-equilibrium-models-of-ai_files/figure-pdf/unnamed-chunk-1-1.pdf}}

\end{footnotesize}}

\begin{description}
\item[Many of the models discussed can be summarized with diagrams.]
Most of the differences between models comes down to assumptions on the
degree of \emph{substitutability} between different inputs, specifically
(1) how substitutable are computers for humans in producing certain
goods; and (2) how substitutable is the demand for different types of
goods. The key substitutability assumptions of most of these models can
be illustrated with a diagram showing networks of inputs and outputs,
here \(\sigma\) represents the \emph{elasticity of substitution}:
\(\sigma=0\) means perfect complements, \(\sigma=\infty\) means perfect
substitutes. I explain below how to interpret the elasticity of
substitution formally.
\item[I also list some things I think are missing from the literature.]
To me the most notable hole in the literature is a theory of what
\emph{types} of tasks AI will be good at. The majority of the papers
discussed assume a single dimension of task ``difficulty'' and represent
AI as continuation of progress in capital's ability to do progressively
more difficult tasks. However it seems clear that recent AI models have
a distinctively different pattern of capabilities than the distribution
across people.
\end{description}

\section{Models}\label{models}

\begin{description}
\item[We will build up a series of models of the impact of AI.]
In this section I start from the simplest model and gradually add
complications with references to individual papers. The following
section lists each of the papers and describes their assumptions and
implications.
\end{description}

\marginnote{\begin{footnotesize}

\pandocbounded{\includegraphics[keepaspectratio]{images/2024-09-06-10-46-02.png}}
In Western countries GDP/capita has increased by about 20 times over
1820-2022
(\href{https://ourworldindata.org/grapher/gdp-per-capita-maddison?tab=chart}{Our
World in Data})

\end{footnotesize}}

\marginnote{\begin{footnotesize}

\pandocbounded{\includegraphics[keepaspectratio]{2024-08-19-equilibrium-models-of-ai_files/figure-pdf/unnamed-chunk-2-1.pdf}}

\end{footnotesize}}

\begin{description}
\item[In the simplest model AI increases output.]
In developed countries the amount of output produced per person has
increased by around 20X over the last 200 years. The simplest model of
AI is that it continues that growth in productivity.

At right we illustrate this with the simplest of diagrams: output is a
function of labor, and we color the arrow from labor to output blue to
represent growth in labor productivity.
\end{description}

\vspace{3cm}

\marginnote{\begin{footnotesize}

\pandocbounded{\includegraphics[keepaspectratio]{images/2024-08-19-14-36-53.png}}
Hours worked per worker fell by about 50\% since 1870. In addition
people now start work at later ages and spend more years in retirement
(\href{https://ourworldindata.org/grapher/gdp-per-capita-maddison\#all-charts}{Our
World in Data})

\end{footnotesize}}

\marginnote{\begin{footnotesize}

\pandocbounded{\includegraphics[keepaspectratio]{2024-08-19-equilibrium-models-of-ai_files/figure-pdf/unnamed-chunk-3-1.pdf}}

\end{footnotesize}}

\begin{description}
\item[AI will reduce hours worked.]
Over the last 100 years output/worker has grown by about 10X in
developed countries, and hours worked have fallen by about 50\%. Thus if
AI increases productivity we could expect that it will decrease the
total hours worked, although by a much smaller proportional amount than
the increase in output.\sidenote{\footnotesize Boppart and Krusell (2020) say ``across
  countries and historically, hours fall steadily by a little below
  0.5\% per year.''}

We can represent the tradeoff between output and leisure in the diagram
at right (leisure is defined as time spent not working). The arrow from
time to output is colored blue to represent an increase in productivity.
The historical trends imply an elasticity of substitution around
0.7.\sidenote{\footnotesize If the relative price of output fell by 10X and the
  relative consumption increased by 5X this implies an elasticity of
  substitution between consumption and leisure slightly below 1
  (\(\sigma=\frac{d\ln(x_G/x_L)}{d\ln(p_L/p_G)}=\frac{\ln(5)}{\ln(10)}\simeq 0.7\)).}
\end{description}

\vspace{3cm}

\marginnote{\begin{footnotesize}

\pandocbounded{\includegraphics[keepaspectratio]{images/2024-08-20-12-43-29.png}}
US employment shares over time
(\href{https://www.clevelandfed.org/publications/economic-commentary/2019/ec-201909-changes-in-us-occupational-structure}{Cleveland
Fed})

\end{footnotesize}}

\marginnote{\begin{footnotesize}

\pandocbounded{\includegraphics[keepaspectratio]{2024-08-19-equilibrium-models-of-ai_files/figure-pdf/unnamed-chunk-4-1.pdf}}

\end{footnotesize}}

\begin{description}
\item[Employment will decline in the sectors where productivity
increases.]
AI is likely to increase productivity (output/hour) much more in some
sectors. This could cause employment to either increase or decrease in
the most-affected sectors. Historically the sectors which have had the
highest growth in productivity have tended to shrink in employment. Over
the last 150 years employment has shifted from agriculture, to
manufacturing, and then to services, roughly in line with productivity
growth in each sector. Put another way, people became satiated with
cheap agricultural goods, and then satiated with cheap manufactured
goods, and so now spend most of their income on services.\sidenote{\footnotesize Nordhaus
  (2021) says ``the sectors that are experiencing the most rapid price
  declines are also experiencing slight declines in expenditure
  shares.'' This is also the assumption underlying the ``cost disease''
  model in Baumol and Bowen (1965).}

This pattern can be represented by the elasticity of substitution. Ngai
and Pissarides (2007) estimate an elasticity of substitution of of 0.6
between agriculture, manufacturing and services from historical changes.

Thus if AI dramatically increases productivity in knowledge work but not
in-person services then we'd expect a reallocation of labor from the
former to the latter.
\end{description}

\vspace{3cm}

\marginnote{\begin{footnotesize}

Acemoglu and Restrepo (2022) find that industries with high rates of
adoption of automation (e.g.~car manufacturing, computer services,
metals, chemicals, plastics) had much lower wage growth, even
controlling for the level of education.

\end{footnotesize}}

\begin{description}
\item[Workers with skills replaced by AI will suffer extended loss of
earnings.]
Labor does not flow perfectly between occupations. Historically we have
seen that people whose skills are replaced by automation have lower
incomes: Acemoglu and Restrepo (2022) estimate that automation had large
negative effects on US workers who had skills that were automated,
Autor, Dorn, and Hanson (2021) estimate that competition from China had
large persistent declines in employment and income for those who worked
in manufacturing.

However different empirical studies find different results:

\begin{itemize}
\tightlist
\item
  Giuntella, Lu, and Wang (2022) finds adoption of robots in China
  reduced employment and wages.
\item
  J. Bessen et al. (2023) finds that automation decreases employment of
  incumbent workers in Holland, but there was no effect of computer
  adoption.
\item
  Humlum (2019) estimates that robot adoption in Denmark increased
  lowered the wages of production workers by 5\% but raised average real
  wages by 0.8\%.
\item
  Adachi, Kawaguchi, and Saito (2024) find that robot adoption in Japan,
  instrumented with robot prices, caused an increase in employment in
  the corresponding sector.
\item
  Hotte, Somers, and Theodorakopoulos (2023) reviews 127 studies on
  technological change and employment.
\end{itemize}

A number of recent papers model frictions that prevent efficient
adjustment: Costinot and Werning (2023), Beraja and Zorzi (2022), Lehr
and Restrepo (2022).
\end{description}

\vspace{3cm}

\marginnote{\begin{footnotesize}

Since 1963 the real hourly wages have grown by 70\% for male workers
with post-college educations, but have been essentially unchanged for
high school dropouts. Note that over the same period there was a large
increase in education which normally would be expected to have a
counterbalanced effect (i.e.~lowering the relative wages of the
educated).

\end{footnotesize}}

\marginnote{\begin{footnotesize}

\pandocbounded{\includegraphics[keepaspectratio]{2024-08-19-equilibrium-models-of-ai_files/figure-pdf/unnamed-chunk-5-1.pdf}}

\end{footnotesize}}

\begin{description}
\item[The relative effect of AI on educated or uneducated workers is
unclear.]
It is hard to say whether AI will have a bigger effect on educated or
uneducated workers.

Over 1980-2000 the wages of college graduates increased dramatically
relative to non-college graduates in the US, despite a big increase in
the number of college graduates over the same period.

A common explanation is that technology disproprtionately increased the
productivity of high-skill labor (people who could use computers), and
that the work of low-skill and high-skill labor are relatively
substitutable (\(\sigma>1\)), thus an educated office worker could now
do work that otherwise would be done by a secretary and clerk. Katz and
Murphy (1992) estimate an elasticity of substitution between college and
high-school labor of 1.4 (Katz and Murphy 1992).
\end{description}

\vspace{1cm}

\marginnote{\begin{footnotesize}

\pandocbounded{\includegraphics[keepaspectratio]{images/2024-08-20-11-54-54.png}}
from ILO (2015).

\end{footnotesize}}

\marginnote{\begin{footnotesize}

\pandocbounded{\includegraphics[keepaspectratio]{2024-08-19-equilibrium-models-of-ai_files/figure-pdf/unnamed-chunk-6-1.pdf}}

\end{footnotesize}}

\begin{description}
\item[The labor share may fall.]
A lot of the discussion about AI focusses on what share of GDP will be
paid to labor (in wages) and what share will be paid to capital. Over
the last 100 years the share of income going to labor in Western
countries has typically hovered around 2/3, but it has declined somewhat
since 1980.\sidenote{\footnotesize Autor et al. (2020) say \emph{``Although there is
  controversy over the degree to which the fall in the labor share of
  GDP is due to measurement issues \ldots{} there is a general consensus
  that the fall is real and significant.''}}

A simple model of the economy is that output is produced by labor and
capital (businesses, land, energy, computers), and each is paid their
marginal product. We can model AI as increasing the effective
productivity of capital inputs (the blue arrow). This will increase the
marginal product of both capital and of labor. If we assume that the
elasticity of substitution is 1 (AKA Cobb-Douglas) this would imply that
the returns to capital and labor increase by equal amounts, and so the
labor share of income remains constant.\sidenote{\footnotesize Gechert et al. (2022)
  survey the literature and argue the elasticity of substitution between
  capital and labor is well below 1, which would imply that increasing
  the productivity of capital would increase the labor share. Acemoglu
  (2024) assumes aggregate capital and labor have elasticity of
  substitution of 0.5.}

The recent decline in the labor share has had a number of explanations,
most prominent is Autor et al. (2020) who note that it is mostly driven
by a reallocation of product market shares towards firms with higher
productivity and lower labor costs, \& they attribute that to
technological changes which have strengthened competition.\sidenote{\footnotesize They
  say '' a rise in superstar firms would occur if consumers have become
  more sensitive to quality-adjusted prices due to, for example, greater
  product market competition (e.g., through globalization) or improved
  search technologies \ldots{} or the growth of platform competition''}
\end{description}

\marginnote{\begin{footnotesize}

\pandocbounded{\includegraphics[keepaspectratio]{2024-08-19-equilibrium-models-of-ai_files/figure-pdf/unnamed-chunk-7-1.pdf}}

\end{footnotesize}}

\begin{description}
\item[Wages will depend on the substitutability across goods.]
We can instead divide the economy into \(N\) different tasks (or
``intermediate goods'') and model the impact of AI as progressively
allowing AI to perform more tasks. The implication for labor income will
depend on the substitutability of the different tasks in utility.

Zeira (1998) assumes each task can be produced either by labor or
capital (i.e.~their elasticity of substitution is \(\infty\)), and over
time we progressively learn how to automate different tasks, illustrated
by the blue arrow connecting capital to task \(n\). They also assume
that the relative productivity of capital and labor between tasks is the
same (we return to this below). If the share of income spent on each
task remains constant (\(\sigma=1\)) then the capital share will be
equal to the fraction of tasks that are automated, and so will steadily
increase with automation.\sidenote{\footnotesize This discussion is taken from Aghion,
  Jones, and Jones (2019) on p4.}

However it is more reasonable to assume that \(\sigma<1\), consistent
with the discussion above on structural change. This implies that, as
each task is automated, the expenditure on that task will fall. When an
additional task is automated there are two effects on labor income: (1)
the price of that good falls, increasing the real wage; (2) the demand
for labor to produce that good falls, decreasing the real wage.

In fact the capital share has stayed mostly constant over the last 150
years despite a great deal of automation. Aghion, Jones, and Jones
(2019) note that the capital share could be stable if (1) the elasticity
of substitution between goods is below 1, i.e.~they are gross
complements (consistent with the evidence on structural change given
above), and (2) the progress of automation is becoming progressively
slower, e.g.~if a constant fraction of the remaining human tasks are
automated each year. Korinek and Suh (2024) make a related argument,
that labor income will increase if automation is sufficiently slow
(there is a ``long tail'' of tasks).
\end{description}

\marginnote{\begin{footnotesize}

In the US the majority of workers in 2018 are in occupations that did
not exist in 1940 (red bars), from Autor et al. (2024).

\end{footnotesize}}

\marginnote{\begin{footnotesize}

\pandocbounded{\includegraphics[keepaspectratio]{2024-08-19-equilibrium-models-of-ai_files/figure-pdf/unnamed-chunk-8-1.pdf}}

\end{footnotesize}}

\begin{description}
\item[AI is likely to create new occupations.]
In addition to shifting people between occupations, technology has
created entirely new occupations: Autor et al. (2024) find that the
majority of US workers in 2018 have job titles that did not exist in the
Census of 1940. Acemoglu and Restrepo (2018) have a task model in which
technological progress extends the abilities of both capital and labor:
it allows capital to perform an additional human-only task \(K\), and it
allows labor to perform an entirely new task \(K\). This can generate a
steady progress of automation while still maintaining a constant labor
share of income.
\end{description}

\vspace{3cm}

\begin{description}
\tightlist
\item[The effect of AGI on wages is likely negative.]
The following points discuss the implications of computers being able to
do all human work, AKA Artificial General Intelligence (AGI). We show
that the implications for wages will depend on the degree of comparative
advantage, and the relative consumption of inputs like land and energy.
\end{description}

\vspace{3cm}

\begin{description}
\item[If we have AGI and no comparative advantage then wages will fall.]
The task-based models discussed so far describe an equilibrium where
only some tasks are automated, and the labor share is maintained either
because the remaining human-done tasks become more valuable (Aghion,
Jones, and Jones (2019)) or technological progress creates new
human-only tasks (Acemoglu and Restrepo (2018)). These assumptions are
necessary to fit historical data.

However it is reasonable to ask what would happen if AI can do
\emph{all} the tasks that humans can do. We first assume that humans and
computers have the same relative costs of doing each task (assumed in
Aghion, Jones, and Jones (2019) and Korinek and Suh (2024)). In this
case humans and computers are essentially interchangeable, and so the
price of computers will rise and the price of humans will fall until
they are the same. Korinek and Suh (2024) note that this will cause
wages to decline in absolute terms: the comparative advantage between
humans and machines has disappeared, so the marginal value of human
labor is what it would be in a world without capital. This will hold
whether capital is fixed or endogenous. However note that despite wages
falling total output will be very large, this is primarily a
distribution problem.
\end{description}

\begin{description}
\item[If humans retain a comparative advantage then wages would remain
high.]
The previous conclusion relied on humans and computers having the same
\emph{relative} costs across tasks. However it's clear that some human
tasks are relatively much easier for a computer, e.g.~doing numerical
calculation, relative to transcribing a recording. This implies that
even if computers can do every task there are differences in comparative
advantage, and thus gains from trade. This argument is made by
(\textbf{Smith2024?}): ``humans will have plentiful, high-paying jobs in
the age of AI dominance.''

Under this assumption achieving AGI will cause high wages, but further
growth in computer capabilities (or further accumulation of capital)
will not increase human wages any further, and therefore the labor share
of output will fall.\sidenote{\footnotesize This is what happens in Acemoglu and
  Restrepo (2018) with ``full automation'': there is comparative
  advantage between labor and capital, but the labor share goes to zero.}
\end{description}

\marginnote{\begin{footnotesize}

\pandocbounded{\includegraphics[keepaspectratio]{2024-08-19-equilibrium-models-of-ai_files/figure-pdf/unnamed-chunk-9-1.pdf}}

\end{footnotesize}}

\begin{description}
\item[If computers and humans compete for other inputs then AGI will
lower wages.]
Suppose as before that computers can do all human tasks, but that they
require energy, space, or materials. Then increasing the productivity of
computers will cause those other factors of production to be shifted
towards computer production, which will lower the marginal product of
labor, and thus will lower wages.

This was essentially the case of horses: they still have a comparative
advantage relative to cars but their input and maintenance costs are
much higher, and for that reason the economic value of horses has fallen
dramatically.\sidenote{\footnotesize (\textbf{Smith2024?}) seems to acknowledge that
  this argument undercuts the comparative advantage argument for high
  wages. He has two responses: (1) that the government would limit
  energy use by AI (``if human lives are at stake \ldots{} most
  governments seem likely to limit AI's ability to hog energy.'') or (2)
  that the non-energy costs of AI compute are high (``turning energy
  into compute is really, really expensive and hard \ldots{} {[}t{]}hose
  bottlenecks are specific to compute; unlike energy, they're not things
  that you can allocate back and forth between compute manufacturing and
  human consumption.'')}

Korinek and Suh (2024) briefly discuss competition for fixed factors. A
NY Times article reports responses from a few economists (Autor,
Acemoglu) to a comparative advantage argument who say that they do not
expect computers to be able to do all human tasks, but if they do then
they do not think wages will hold up because of the resource costs.
Autor: ``humans have a real cost of upkeep,\ldots{} Accordingly, humans
might become a noncompetitive factor of production for any
activity.''\sidenote{\footnotesize \href{https://www.nytimes.com/2024/03/22/opinion/ai-jobs-comparative-advantage.html}{Peter
  Coy, NY Times March 22 2024}.}
\end{description}

\marginnote{\begin{footnotesize}

\pandocbounded{\includegraphics[keepaspectratio]{2024-08-19-equilibrium-models-of-ai_files/figure-pdf/unnamed-chunk-10-1.pdf}}

\end{footnotesize}}

\begin{description}
\item[Knowledge sharing: wages will level up.]
This is a model that I have not seen in the literature but it seems
natural way of representing how LLMs are being used in
practice.\sidenote{\footnotesize It's closely related to an informal argument in Autor
  (2024): he says ``{[}AI{]} would simultaneously temper the monopoly
  power that doctors hold over medical care, lawyers over document
  production, software engineers over computer code, professors over
  undergraduate education, etc.''}. Suppose each person's comparative
advantage is constituted by their private knowledge, but LLMs make that
private knowledge public. Now the doctor can be a proficient lawyer, and
vice versa. This has a number of implications:

\begin{enumerate}
\def\labelenumi{\arabic{enumi}.}
\tightlist
\item
  Overall output will increase, especially in high-paid professions
  where knowledge is scarce (medicine, engineering, law), and wages will
  flatten.
\end{enumerate}

\begin{enumerate}
\def\labelenumi{\arabic{enumi}.}
\setcounter{enumi}{1}
\tightlist
\item
  Measured GDP is likely to fall because exchange of services will be
  less necessary, as knowledge is shared.
\end{enumerate}

\begin{enumerate}
\def\labelenumi{\arabic{enumi}.}
\setcounter{enumi}{2}
\tightlist
\item
  The incentives to discover new information will weaken, because there
  will be less return from possessing private information.
\end{enumerate}
\item[AI contributions to productivity growth.]
A number of recent papers discuss AI's ability to directly contribute to
innovation: Aghion, Jones, and Jones (2019), Nordhaus (2021), Davidson
(2021), Erdil and Besiroglu (2023), Trammell and Korinek (2023).
\item[AI and directed technical change.]
A number of papers discuss the ability to direct technical change, e.g.
Acemoglu and Restrepo (2018) have a model in which innovation can either
create new tasks for computers or new tasks for humans. Brynjolfsson
(2023) and Autor (2024) discuss in an informal way the potential for
human-augmenting vs human-automating AI innovation.
\item[Asset pricing.]
A number of recent papers discuss how AI will be reflected in asset
prices. Chow, Halperin, and Mazlish (2023) argue that expectations of AI
should increase real interest rates through both (1) expectations of
higher future consumption, and (2) concerns about existential risk.
However they find little evidence that interest rates have increased.
Eisfeldt et al. (2023) find that, on the release of ChatGPT, the value
of firms with higher share of workers that are ``exposed'' to AI
relatively increased.
\end{description}

\section{Models of Automation and AI
(Chronological)}\label{models-of-automation-and-ai-chronological}

\marginnote{\begin{footnotesize}

\pandocbounded{\includegraphics[keepaspectratio]{2024-08-19-equilibrium-models-of-ai_files/figure-pdf/unnamed-chunk-11-1.pdf}}

\end{footnotesize}}

\begin{description}
\item[Zeira (1998) ``Workers, Machines, and Economic Growth'']
Zeira (1998) divides the economy into \(N\) intermediate goods, each of
which can be produced by labor, and some can additionally be produced by
capital. Assuming that all the goods have elasticity of substitution of
1 means the capital share of the economy will be exactly the share of
goods that can be produced by capital.\sidenote{\footnotesize This discussion is taken
  from Aghion, Jones, and Jones (2019) on p4.}
\end{description}

\marginnote{\begin{footnotesize}

\pandocbounded{\includegraphics[keepaspectratio]{2024-08-19-equilibrium-models-of-ai_files/figure-pdf/unnamed-chunk-12-1.pdf}}

\end{footnotesize}}

\begin{description}
\item[Krusell et al. (2000), ``Capital-skill complementarity and
inequality: A macroeconomic analysis'']
They argue that the rise in the skill premium can mainly be explained by
complementarity between capital and skill, rather than by a change in
technology that increases the returns to skill.

They construct a nested production function with four inputs (capital
equipment, capital structures, skilled labor, and unskilled labor), and
they estimate elasticities of substitution between each of the inputs.
\item[Autor, Levy, and Murnane (2003), ``The Skill Content of Recent
Technological Change: An Empirical Exploration'']
They split labor into routine tasks (operate tabulating machine, pack
goods) and non-routine tasks (design a building, navigate a boat), with
substitutability index of 1. They assume that computers are perfect
substitutes for routine tasks: this will lower the returns to routine
labor, and raise the return to non-routine labor. Each person has a
certain productivity in routine vs non-routine tasks. As computer
capital grows this increases the wages of non-routine workers, and
decreases the wages of routine workers. They say \emph{``the model can
explain 60 percent of the estimated relative demand shift favoring
college labor during 1970 to 1998. Task changes within nominally
identical occupations account for almost half of this impact.''}

Routine: \emph{``can be accomplished by following explicit rules''};
non-routine: \emph{``problem-solving and complex communication
activities.''} They use various measures from DOT to define
routine/nonroutine and analytic/manual (p1293). For routine cognitive:
``STS, which measures adaptability to work requiring Set limits,
Tolerances, or Standards.''

Professions that have been mostly eliminated: elevator operator,
switchboard operator, typist, assembly line worker, data entry clerk,
toll booth operator, film projectionist, mail sorter, meter reader,
stock broker, parking attendant, printer, bookkeeper, order clerk.

\[\begin{aligned}
  Q &=(L_R+C)^{1-\beta}L_N^\beta\\
  C &= \text{computer}\\
  L_R &= \text{routine labor}\\
  L_N &= \text{non-routine labor}
   \end{aligned}\]
\end{description}

\marginnote{\begin{footnotesize}

\pandocbounded{\includegraphics[keepaspectratio]{2024-08-19-equilibrium-models-of-ai_files/figure-pdf/unnamed-chunk-13-1.pdf}}

\end{footnotesize}}

\begin{description}
\item[Sachs and Kotlikoff (2012) ``Smart Machines and Long-Term
Misery'']
They assume machines are relatively more substitutable with unskilled
than skilled labor, by them both producing intermediate goods. They note
that an increase in machine productivity will lower unskilled wages if
(1) substitutability of machines and unskilled labor is high
(\(\varepsilon_{ML}\) large); (2) substitutability of intermediate goods
and skilled labor is low (\(\varepsilon_{SN}\) small); and (3) the share
of skilled labor in final output is high.

They then build an overlapping generations model, where each agent is
unskilled in the first period, and can save to invest in a mixture of
machinery and skills. They show that a rise in machine productivity can
permanently lower the welfare of all succeeding generations through
lowering their ability to save. Sachs has a number of other papers with
other coauthors that appear to extend this dynamic framework.
\item[Aghion, Jones, and Jones (2019) ``Artificial Intelligence and
Economic Growth'']
This paper discusses the task-based model of Zeira (1998), which implies
that automation should cause a progressive increase in the labor share.
They show that automation can generate balanced growth (meaning the
labor share asymptotes to a constant level) if the elasticity of
substitution between capital and labor is below 1, and the automation
process is such that constant \emph{fraction} of the remaining human
tasks are automated each year.

They also discuss a model with both production and research, where both
are Cobb-Douglas, in which growth will be hyperbolic.
\end{description}

\marginnote{\begin{footnotesize}

\pandocbounded{\includegraphics[keepaspectratio]{2024-08-19-equilibrium-models-of-ai_files/figure-pdf/unnamed-chunk-14-1.pdf}}

\end{footnotesize}}

\begin{description}
\tightlist
\item[J. E. Bessen (2016), ``How computer automation affects
occupations: Technology, jobs, and skills'']
This paper has a model in which there are different types of labor
(plumber, lawyer), and each occupation consists of a bundle of tasks.
Automation typically automates just one task, which . Bessen gives
examples: introduction of the ATM did not lead to fewer bank tellers,
introduction of the barcode scanner did not reduce the number of
checkout clerks. He says that there are some occupations that have
declined due to automation (telephone operators, typesetters), but most
occupations that are exposed to automation have increased. He disagrees
with Autor, Levy, and Murnane (2003)'s interpretation that computers led
to a hollowing out.
\item[Acemoglu and Restrepo (2018) ``The Race Between Man and Machine:
Implications of Technology for Growth, Factor Shares, and Employment'']
This model is similar to task models discussed above, but additionally
assumes that techincal progress creates new tasks that only humans can
do, e.g.~``radiology technician'' and ``management analyst''.

The model thus generates the observed historical pattern that (a) we are
continually automating human tasks, yet (b) the share of labor in output
has remained about the same. There's a continuum of tasks from \(N-1\)
to \(N\), labor can do all of them but capital can only do some of them,
and among those which capital can do, labor has a comparative advantage
in tasks with a higher index. Advance of technology means that capital
can do some new tasks, but we also create new labor-only tasks (increase
\(N\)).\sidenote{\footnotesize A nice observation (p1500): if the CES coefficients
  (\(\sigma\) and \(\zeta\)) are both equal to 1 then aggregate output
  will be Cobb-Douglas, where the capital share is the share of tasks
  that are automated.}

Note: they discuss a scenario in which there is full automation
(Proposition 4 (i) and Proposition 6 (i)), and say that all tasks will
be done by capital, and the labor share will go to zero, but it is
unclear to me why the comparative advantage between labor and capital
does not hold. Also it is notable that they do not discuss competition
for fixed factors if there is full automation (land and energy).
\end{description}

\marginnote{\begin{footnotesize}

Output is a CES aggregate of tasks:
\[Y=\tilde{B} \left[ \int_{N-1}^N y(i)^{\frac{\sigma-1}{\sigma}} \, d\Phi(i) \right]^{\frac{\sigma}{\sigma-1}}\]
Each task is a CES aggregate of labor and task-specific technology
\(q(i)\): \[y(i)=\begin{cases}
      \overline{B}(\zeta) \left[ \eta^{\frac{1}{\zeta}} q(i)^{\frac{\zeta - 1}{\zeta}} + (1-\eta)^{\frac{1}{\zeta}} \left( k(i) + \gamma(i) l(i) \right)^{\frac{\zeta - 1}{\zeta}} \right]^{\frac{\zeta}{\zeta - 1}}
      &, i\leq I\\
      \overline{B}(\zeta) \left[ \eta^{\frac{1}{\zeta}} q(i)^{\frac{\zeta - 1}{\zeta}} + (1-\eta)^{\frac{1}{\zeta}} \left( \gamma(i) l(i) \right)^{\frac{\zeta - 1}{\zeta}} \right]^{\frac{\zeta}{\zeta - 1}}
      &, i>I
   \end{cases}
\]

\end{footnotesize}}

\begin{description}
\tightlist
\item[Nordhaus (2021), ``Are we Approaching an Economic Singularity?
Information Technology and the Future of Economic Growth.'']
Nordhaus discusses ways in which AI could accelerate economic growth. He
says the recent trends in growth do not make it look like it's imminent.
Particular points:

\begin{enumerate}
\def\labelenumi{(\arabic{enumi})}
\tightlist
\item
  Historically sectors with higher growth rates have declined as a share
  of output, implying that acceleration in individual areas will not
  lead to topline acceleration in output growth.
\end{enumerate}

\begin{enumerate}
\def\labelenumi{(\arabic{enumi})}
\setcounter{enumi}{1}
\tightlist
\item
  Recent macroeconomic trends imply that capital is only slowly
  substituting for labor. The singularity would happen when they are
  perfect substitutes, and extrapolating recent trends does not imply
  that they are close to becoming perfect substitutes.
\end{enumerate}
\item[Hemous and Olsen (2022), ``The Rise of the Machines: Automation,
Horizontal Innovation, and Income Inequality'']
I believe this paper primarily expands previous models by endogenizing
the direction of technical change.

There are a set of \(N\) intermediate goods with some elasticity of
substitution. Each good is produced from high-skilled and low-skilled
labor with Cobb-Douglas production, however some sectors can use
machines to substitute for low-skilled labor (p.~183, eq. 2). Over time
the set of goods increases (\(N\)). There is no capital, instead
machines are themselves produced from the final good.
\end{description}

\marginnote{\begin{footnotesize}

\pandocbounded{\includegraphics[keepaspectratio]{2024-08-19-equilibrium-models-of-ai_files/figure-pdf/unnamed-chunk-15-1.pdf}}

\end{footnotesize}}

\begin{description}
\tightlist
\item[Acemoglu and Restrepo (2022), ``Tasks, Automation, and the Rise in
in US Wage Inequality'']
Each type of worker does the task in which they have a comparative
advantage. Gradually computers get better at different tasks, and then
take over that task. Each progression increases overall output but when
a group of workers gets displaced then their income declines. They argue
that this model is a good fit for overall wage dynamics over 1980-2020,
as workers in routine industries have become displaced.
\end{description}

\marginnote{\begin{footnotesize}

Parameters (p2008):

\begin{longtable}[]{@{}
  >{\raggedright\arraybackslash}p{(\linewidth - 6\tabcolsep) * \real{0.4122}}
  >{\raggedright\arraybackslash}p{(\linewidth - 6\tabcolsep) * \real{0.0382}}
  >{\raggedright\arraybackslash}p{(\linewidth - 6\tabcolsep) * \real{0.0840}}
  >{\raggedright\arraybackslash}p{(\linewidth - 6\tabcolsep) * \real{0.4656}}@{}}
\toprule\noalign{}
\endhead
\bottomrule\noalign{}
\endlastfoot
Elasticity of substitution between tasks & 0.5 & \(\lambda\) & (Humlum
2019) \\
Elasticity of substitution between labor and capital & 1 & \(\sigma\) &
Karabarbounis and Neiman (2014), Oberfield and Raval (2021) \\
Elasticity of substitution between industries & 0.2 & \(\eta\) & Buera,
Kaboski, and Rogerson (2015) \\
Cost savings from automation & 30\% & \(\pi\) & Acemoglu and Restrepo
(2020) \\
\end{longtable}

\end{footnotesize}}

\begin{description}
\tightlist
\item[Benzell, Brynjolfsson, and Saint-Jacques (2022) ``Digital
Abundance Meets Scarce Architects'' {[}UNFINISHED{]}]
They have three inputs: capital, labor, and architecture (produced by
labor). There's some elasticity of substitution between regular
capital-labor production and architecture. They say this generates a
decrease in labor share and in interest rates after a technology
advance. They say the returns to architecture come through intangibles
or executive compensation (I think).
\end{description}

\marginnote{\begin{footnotesize}

\pandocbounded{\includegraphics[keepaspectratio]{2024-08-19-equilibrium-models-of-ai_files/figure-pdf/unnamed-chunk-16-1.pdf}}

\end{footnotesize}}

\begin{description}
\tightlist
\item[Trammell and Korinek (2023) ``Economic Growth Under Transformative
AI'' {[}UNFINISHED{]}]
They consider a few mechanisms: (1) AI increases capital's substitution
for labor; (2) AI increases the growth rate of TFP.
\end{description}

\vspace{5cm}

\begin{description}
\tightlist
\item[A. K. Agrawal, Gans, and Goldfarb (2023) ``The Turing
Transformation''\sidenote{\footnotesize I believe A. Agrawal, Gans, and Goldfarb
  (2023) has the same basic argument but I could not find a copy online.}]
They assume that products need a skilled and unskilled input which are
perfect complements (\(\sigma=0\)). If AI can perfectly replace the
skilled labor this will lower the wages of skilled workers but raise the
wages of unskilled workers.

This essay is written in response to Brynjolfsson (``The Turing Trap'')
and Acemoglu, who both wrote pieces saying we should avoid building AI
which directly replaces human skills. They say ``one person's substitute
is another person's complement'', \& this model shows that replacing
some human jobs will raise wages of other humans.

They give some examples: (1) digital maps allow anyone to be a taxi
driver; (2) LLMs raise the productivity of untrained call center
workers; (3) AI diagnosis would let nurses look after patients; (4) AI
translation allows people to transact across borders; (5) LLMs allow
people with poor literacy to communicate well.
\end{description}

\marginnote{\begin{footnotesize}

\pandocbounded{\includegraphics[keepaspectratio]{2024-08-19-equilibrium-models-of-ai_files/figure-pdf/unnamed-chunk-17-1.pdf}}

\end{footnotesize}}

\marginnote{\begin{footnotesize}

\begin{longtable}[]{@{}lll@{}}
\toprule\noalign{}
\endhead
\bottomrule\noalign{}
\endlastfoot
elasticity of substitution between tasks & \(\sigma\) & 0.5 \\
current fraction of automated tasks & \(\Phi_0\) & 0.6 \\
\end{longtable}

\end{footnotesize}}

\marginnote{\begin{footnotesize}

\pandocbounded{\includegraphics[keepaspectratio]{images/2024-08-20-12-30-22.png}}
They predict that the real compensation of labor will increase then
decrease as AI advances.

\end{footnotesize}}

\begin{description}
\item[Korinek and Suh (2024) ``Scenarios for the Transition to AGI'']
As in Zeira (1998) there are a fixed set of tasks performed by labor,
and as AI progresses it unlocks the ability for capital to perform that
task. They assume that the tasks are gross complements (\(\sigma<0.5\)),
meaning that when a task becomes automatable then its expenditure share
will fall.

Their key finding: \emph{``for low levels of automation, advances in
automation increase wages as the economy becomes more productive, but
for higher levels of automation, wages decline due to the displacement
of labor.''}

We can describe the intuition in terms of the gains from \emph{trading}
with another person: the gains are proportional to how different their
skills are (comparative advantage). Korinek and Suh assume that AI's
skills start out very different (they can only do a few tasks), and then
gradually become more similar to human skills, meaning that the relative
productivity across tasks eventually becomes the same.

\[\begin{aligned}
  Y &= A \left[ \int_i y(i)^{\frac{\sigma-1}{\sigma}} \, d\Phi(i) \right]^{\frac{\sigma}{\sigma-1}}
     && \text{(output is CES across tasks)}\\
  y(i) &= a_K(i)k(i) + a_L(i)\ell(i)
     && \text{(each task can be performed by capital or labor)}\\
  a_L(i) &= 1 && \text{(normalize labor productivity to 1)}\\
  a_K(i) &= \begin{cases}1&i<I\\0&i\geq I\end{cases} && \text{(capital gradually catching up)}\\
   \end{aligned}\]
\item[Autor et al. (2024), ``New frontiers: The origins and content of
new work, 1940--2018'']
This paper has a model of rolling tasks which extends Acemoglu and
Restrepo (2018) to two sectors, high-skilled and low-skilled. The model
distinguishes between two types of innovation: augmentation (increases
the value of outputs), and automation (substitutes for labor inputs).
They collect a data about occupations over time. They estimate the
degree of automating vs augmenting innovation by looking at patents
which can be matched to occupational titles (from the Labor Dept's
\emph{Dictionary of Occupational Titles}) vs those that can be matched
to names of industries (from the Census Bureau's \emph{Index of
Occupations and Industries}).

They find that the majority of current employment is in professions that
were only invented since 1940, and they estimate the relative impact of
augmenting vs automating innovation.
\item[Acemoglu (2024) ``The Simple Macroeconomics of AI'']
He sets up a standard task-based model then discusses various different
ways in which AI can change the parameters. However his primary
quantitative prediction over the next decade uses a very simple
calculation, using the following calculation:
\end{description}

\begin{longtable}[]{@{}
  >{\raggedright\arraybackslash}p{(\linewidth - 4\tabcolsep) * \real{0.5304}}
  >{\raggedleft\arraybackslash}p{(\linewidth - 4\tabcolsep) * \real{0.0435}}
  >{\raggedright\arraybackslash}p{(\linewidth - 4\tabcolsep) * \real{0.4261}}@{}}
\toprule\noalign{}
\endhead
\bottomrule\noalign{}
\endlastfoot
Share of tasks exposed to AI & 20\% & Eloundou \\
Share of exposed tasks that can profitably be performed by AI & 23\% &
Svanberg et al. \\
=\textgreater{} GDP share impacted by AI & 4.6\% & \\
& & \\
Labor cost savings in AI-performed tasks & 27\% & Noy and Zhang (2023)
and Brynjolfsson, Li, and Raymond (2023) \\
Labor share of output & 57\% & \\
=\textgreater{} Avg total cost savings & 15.4\% & \\
& & \\
=\textgreater{} Aggregate TFP increase (=GDP share * avg cost savings) &
0.71\% & \\
\end{longtable}

\begin{description}
\item[Aghion and Bunel (2024), ``AI and Growth: where do we stand'']
They use two different methods to estimate the productivity impact, both
yield roughly 1\%/year. Their second method is basically the same as
that used in Acemoglu (2024), but paramerized with much higher numbers,
that seem more reasonable to me.
\item[Autor (2024) ``Applying AI to Rebuild Middle-Class Jobs'']
This essay argues that expertise is the primary source of labor's value,
and that AI can be a complement to expertise rather than a substitute.

\begin{quote}
``Expertise is the primary source of labor's value in the US and other
industrialized countries. Jobs that require little training or
certification, such as restaurant servers, janitors, manual laborers and
(even) childcare workers, are typically found at the bottom of the wage
ladder.''
\end{quote}

He says computers and the internet increased inequality:

\begin{quote}
``by making information and calculation cheap and abundant,
computerization catalyzed an unprecedented concentration of
decision-making power, and accompanying resources, among elite
experts.''
\end{quote}
\end{description}

\marginnote{\begin{footnotesize}

\pandocbounded{\includegraphics[keepaspectratio]{2024-08-19-equilibrium-models-of-ai_files/figure-pdf/unnamed-chunk-18-1.pdf}}

\end{footnotesize}}

\marginnote{\begin{footnotesize}

\pandocbounded{\includegraphics[keepaspectratio]{2024-08-19-equilibrium-models-of-ai_files/figure-pdf/unnamed-chunk-19-1.pdf}}

\end{footnotesize}}

\begin{description}
\item[Ide and Talamas (2024a) ``Artificial Intelligence in the Knowledge
Economy'']
Each human has a \emph{ceiling} on their ability (on a continuum of
tasks from 0 to 1), but there's no comparative advantage below that
ceiling. Thus there's assortative matching between skills and tasks.

The paper has two big predictions:

\begin{enumerate}
\def\labelenumi{\arabic{enumi}.}
\tightlist
\item
  \emph{Non-autonomous AI will benefit the least knowledgeable.} For
  non-autonomous AI you still need a human for every problem, so it
  effectively extends the range of tasks each person can handle.
\end{enumerate}

\begin{enumerate}
\def\labelenumi{\arabic{enumi}.}
\setcounter{enumi}{1}
\tightlist
\item
  \emph{Autonomous AI will benefit the most knowledgeable individuals.}
  Autonomous AI is a substitute for lowexactly like a human, and there's
  vertical strafication, so it'll lower relative returns to low types
  (but assumption of comparative advantage).
\end{enumerate}

Additionally there's friction to trade, so endogenous formation of firms
(partnerships between low \& high ability).
\item[Ide and Talamas (2024b)]
asdf
\end{description}

\begin{description}
\item[Barnett (2025) ``The economic consequences of automating remote
work'']
This blog post estimates that about 1/3 of all labor in the US could be
done remotely. He then asks what would happen if we expand the supply of
remote labor by 100X through AI. If the elasticity of substitution
between remote labor and other labor is 0.5, then output would double.
If the elasticity was 10 (his preferred estimate) then output would
increase by 50X.
\end{description}

\section{Related Models}\label{related-models}

These are not models of AI and automation, however they're useful
reference models.

\marginnote{\begin{footnotesize}

\pandocbounded{\includegraphics[keepaspectratio]{2024-08-19-equilibrium-models-of-ai_files/figure-pdf/unnamed-chunk-20-1.pdf}}

\end{footnotesize}}

\begin{description}
\item[Ricardo (1821) ``Principles of Political Economy''
{[}UNFINISHED{]}]
This is a famous model of trade: suppose English and Portugese labor can
produce cloth and wine with linear production functions, what will
happen if Portugese labor becomes better at producing cloth?

Generally: if your industry becomes more productive, and you are a small
share of output, and demand is inelastic (gross complements), this is
\emph{bad} for you.

If Leontief preferences then prices will be determined by whoever's
budget constraint is the largest, so if you grow productivity then you
grow income, but at some point you discretely become price-setter and so
lose.
\end{description}

\marginnote{\begin{footnotesize}

\pandocbounded{\includegraphics[keepaspectratio]{2024-08-19-equilibrium-models-of-ai_files/figure-pdf/unnamed-chunk-21-1.pdf}}

\end{footnotesize}}

\begin{description}
\tightlist
\item[Dixit and Stiglitz (1977) ``Monopolistic Competition and Optimum
Product Diversity'']
In this paper each type of good requires a fixed cost to produce, and
then is linear in inputs. Here I've put an ``F'' on the side of each
product to indicate a fixed cost. Elasticity of substitution is
typically assumed to be above 1 (\(\sigma>1\)). The important
implication is that as market size gets larger then the number of
products offered will be larger. The model is widely used in macro and
trade.
\end{description}

\marginnote{\begin{footnotesize}

\pandocbounded{\includegraphics[keepaspectratio]{2024-08-19-equilibrium-models-of-ai_files/figure-pdf/unnamed-chunk-22-1.pdf}}

\end{footnotesize}}

\begin{description}
\item[Capital Accumulation: if \(\sigma=1\) then no change.]
We can also model the choice of capital, assuming that labor can be
split between output and maintaining capital goods, and we assume that
the total stock of capital is proportional to the time spent maintaining
capital goods. We can ask what happens when capital productivity
increases:

\begin{itemize}
\tightlist
\item
  If capital and labor have substitutability of 1 (\(\sigma=1\)) then
  capital stock will not change. increases in capital productivity will
  increase output and will increase the returns to both factors, but
  will not change the quantity of capital accumulated, because the
  relative returns to capital and labor do not change (the value of
  capital measured in output will increase, but the physical quantity
  will not change).
\end{itemize}

\begin{itemize}
\tightlist
\item
  If capital and labor are gross substitutes (\(\sigma>1\)) then capital
  will increase. The effective price of capital decreases, and so the
  share of output spent on capital will increase, and so the share of
  time allocated to capital maintenance will increase, and so the
  physical quantity of capital will increase.
\end{itemize}

\begin{itemize}
\tightlist
\item
  If capital and labor are gross complements (\(\sigma<1\)) then capital
  will decrease.
\end{itemize}

An analogy: imagine crop yield depends on harvesting and on scythes, and
that time can be spent on either harvesting or scythe production. If
time and scythes are complements then an increase in scythe productivity
will cause a decrease in the quantity of scythes produced. If time and
scythes are substitutes then more productive scythes will cause an
increase in the quantity produced.
\item[Autor et al. (2020) ``The Fall of the Labor Share and the Rise of
Superstar Firms'']
They assume there are a variety of firms, with the same Cobb-Douglas
production function, but each has a different level of total factor
productivity. They assume that higher productivity firms have higher
proportional markups, which will hold for certain utility functions
(however not with a CES utility function which implies fixed
proportional markups).

They then empirically show that there has been a lot of reallocation
towards higher-productivity firms, which can explain the increasing
trends in markups and declining labor share.
\item[Jones (2023) ``The AI dilemma: Growth versus existential risk'']
This paper makes two points about the tradeoff between AI-driven
economic growth and AI-caused existential risk:

\begin{enumerate}
\def\labelenumi{\arabic{enumi}.}
\tightlist
\item
  The appropriate tradeoff is extremely sensitive to the curvature of
  utility: if we have log utility then we would be prepared to risk a
  lot; if we have CRRA bounded utility then we woudl take very little
  risk.
\end{enumerate}

\begin{enumerate}
\def\labelenumi{\arabic{enumi}.}
\setcounter{enumi}{1}
\tightlist
\item
  If AI growth delivers greater life expectancy or population then the.
\end{enumerate}
\end{description}

\section{Technical Details}\label{technical-details}

\marginnote{\begin{footnotesize}

\pandocbounded{\includegraphics[keepaspectratio]{2024-08-19-equilibrium-models-of-ai_files/figure-pdf/unnamed-chunk-23-1.pdf}}

\end{footnotesize}}

\begin{description}
\item[Macroeconomic models are built from a chain of functions.]
Macroeconomic models are typically built from production functions
(mapping inputs to goods), and utility functions (mapping goods to
consumer value). We'll represent these functions with diagrams like
those on the right.
\item[The critical assumptions in a model are over the substitutability
of inputs (\(\sigma\)).]
Most of the difference between models is in the assumptions about
\emph{substitutability} of inputs, represented with \(\sigma\). The
elasticity can be defined as the effect of a change in relative price on
relative consumption (assuming \(x_1\) and \(x_2\) are chosen to
maximize profits given input prices \(p_1\) and \(p_2\)):
\[\sigma_{21}=\frac{d\ln(x_1/x_2)}{d\ln(p_1/p_2)}.\]

It is often convenient to assume that functions have the following
structure, in which the elasticity of substitution is equal to
\(\sigma\) everywhere (i.e.~for every value of the inputs, \(A,B,C\)).
\[\begin{aligned}
  \text{output}&=\left(\text{A}^\frac{\sigma}{\sigma-1}+\text{B}^\frac{\sigma}{\sigma-1}+\text{C}^\frac{\sigma}{\sigma-1}\right)^\frac{\sigma-1}{\sigma}.
   \end{aligned}\]
\end{description}

\marginnote{\begin{footnotesize}

\pandocbounded{\includegraphics[keepaspectratio]{2024-08-19-equilibrium-models-of-ai_files/figure-pdf/unnamed-chunk-24-1.pdf}}

\end{footnotesize}}

\begin{description}
\tightlist
\item[When \(\sigma>1\), inputs are substitutes.]
For example tractors and horses are substitutes in producing wheat. This
has two useful implications: (1) when the quantity of horses increases,
the share of income going to horses will increase (assuming each input
is paid their marginal product); (2) when the price of tractors falls
then the quantity of horses demanded will fall.
\end{description}

\marginnote{\begin{footnotesize}

\pandocbounded{\includegraphics[keepaspectratio]{2024-08-19-equilibrium-models-of-ai_files/figure-pdf/unnamed-chunk-25-1.pdf}}

\end{footnotesize}}

\begin{description}
\tightlist
\item[When \(\sigma<1\), inputs are complements.]
For example cowboys and horses are complements in producing beef. This
has two implications: (1) when the quantity of horses increases the
share of income going to horses will decrease; (2) when the price of
horses falls then the quantity of cowboys demanded will increase (for a
given price of cowboys).
\end{description}

\marginnote{\begin{footnotesize}

\pandocbounded{\includegraphics[keepaspectratio]{2024-08-19-equilibrium-models-of-ai_files/figure-pdf/unnamed-chunk-26-1.pdf}}

\end{footnotesize}}

\begin{description}
\tightlist
\item[When \(\sigma=1\), factor shares will be fixed.]
For example suppose we model a strawberry farm as producing
strawberries, with land and farmworkers as inputs.
\end{description}

\marginnote{\begin{footnotesize}

\pandocbounded{\includegraphics[keepaspectratio]{2024-08-19-equilibrium-models-of-ai_files/figure-pdf/unnamed-chunk-27-1.pdf}}

\end{footnotesize}}

\begin{description}
\tightlist
\item[Many of the models have two stages.]
In the first stage factors combine to create goods, in the second stage
goods combine to create utility.
\end{description}

\marginnote{\begin{footnotesize}

\[\begin{aligned}
   good_j &= \left(\sum_{i}factor_{i,j}^\frac{\sigma_j}{\sigma_j-1}\right)^\frac{\sigma_j-1}{\sigma_j}
      && \text{(production)} \\
   u&=\left(\sum_{j}good_j^\frac{\sigma}{\sigma-1}\right)^\frac{\sigma-1}{\sigma}
      && \text{(utility)} \\
\end{aligned}
\]

\end{footnotesize}}

\begin{description}
\item[We assume that all consumers consume the same ratio of goods.]
We assume that utility is homothetic meaning that the relative
consumption of different goods is independent of the level of income.
Although homotheticity certainly isn't exactly true (e.g.~Engel curves
for food), Ngai and Pissarides (2007) argue that it's a reasonable
approximation and is consistent with the broad facts of structural
change.
\item[A small changes to productivity will be directly reflected in
output.]
Suppose productivity grows by 20\% in one sector, and that sector
constitutes 10\% of the economy, then we should expect total output to
increase by 2\% (i.e.~20\% x 10\%). This is Hulten's theorem (Hulten
1978), and holds for small changes in a competitive economy with
constant returns to scale (see derivation in Acemoglu (2024)).
\item[A large change in productivity will depend on the degree of
complementarity.]
Suppose AI makes labor much more productive in some sector, e.g.~10X
more productive, then the effect on output will depend on
substitutability. If \(\sigma=1\) (Cobb-Douglas), then Hulten's theorem
will continue to be accurate for large changes, but if \(\sigma<0\)
(gross complements) then the effect will be smaller, where \(\lambda\)
is the growth in productivity, and \(X\) is the factor share:

\[\utt{\frac{Y'}{Y}}{output}{growth}=[(1-X)+ X \lambda^{-(1-\sigma)} ]^{-\frac{1}{1-\sigma}}\]

If \(\lambda=\infty\) then growth is \((1-X)^{-\frac{1}{1-\sigma}}\).

If perfect complements (\(\sigma=0\)) then
\(\frac{Y'}{Y}=\frac{1}{1-X}\), thus automating 10\% of the economy will
get 10\% growth, and automating 50\% of the economy would get 100\%
growth.

If perfect complements (\(\sigma=0\)) then
\(\frac{Y'}{Y}=[1-X(1-\lambda^{-1})]^{-1}\).
\end{description}

\sidenote{\footnotesize Baqaee and Farhi (2019) extend Hulten's theorem beyond
  first-order terms and show that second-order terms are often
  important, e.g.~in calibrations to oil shocks .}

\begin{description}
\item[The effect of productivity on factor income depends on
substitutability and the factor share.]
Suppose the productivity of factor \(X\) increases: this will increase
overall output, and the incomes of all other factors, but if
\(\sigma<1\) then it will decrease the \emph{share} of output going to
factor \(X\). The net effect on factor income will be positive if and
only if the share of output going to that factor, \(s_X\), starts at a
sufficiently high level \(s_X>\sigma\) (see Acemoglu (2024) p12).
\end{description}

\subsection{A Knowledge-Sharing Model}\label{a-knowledge-sharing-model}

See also \texttt{2024-08-05-ricardian-ai.md}

\marginnote{\begin{footnotesize}

\pandocbounded{\includegraphics[keepaspectratio]{2024-08-19-equilibrium-models-of-ai_files/figure-pdf/unnamed-chunk-28-1.pdf}}

\end{footnotesize}}

Suppose we think of LLMs as making private knowledge public. We start
with a world where each person has an absolute advantage in the output
that they produce:

Then LLMs share knowledge which makes everyone better at \emph{every}
tasks, i.e.~tasks that they're not already specialized in.

\marginnote{\begin{footnotesize}

\pandocbounded{\includegraphics[keepaspectratio]{2024-08-19-equilibrium-models-of-ai_files/figure-pdf/unnamed-chunk-29-1.pdf}}

\end{footnotesize}}

Implications \& observations:

\begin{enumerate}
\def\labelenumi{\arabic{enumi}.}
\item
  \textbf{The composition of output will shift.} We will produce
  relatively much more of the things where knowledge is a scarce input
  (and so wages are high): medicine, engineering, law.
\item
  \textbf{Highly paid professiosn wages will fall} The quantity of
  output will increase and the wage will fall, but aggregate welfare
  increases.
\item
  \textbf{Measured GDP could fall.} Because exchange of services would
  be less necessary, because we can do those things ourselves.
\item
  \textbf{The incentives to discover new information will weaken.} If
  you expect any private information to become public, through LLMs,
  then you will no longer be able to earn a return from possessing that
  information.
\item
  \textbf{Junior workers become experts / learning curve is compressed.}
  Junior workers can use LLMs to very quickly catch up.
\item
  \textbf{Even physical jobs would be affected.} Strawberry pickers,
  baristas, arborists, can ask an LLM how to solve a problem, or can ask
  multimodal model to critique their form.
\item
  \textbf{This is similar to books and the internet.} The printing press
  and the internet also dramatically changed the distribution of
  information, I guess they'd have comparable economic impact.
\end{enumerate}

\newpage

\newpage

\section*{Bibliography}\label{bibliography}
\addcontentsline{toc}{section}{Bibliography}

\phantomsection\label{refs}
\begin{CSLReferences}{1}{0}
\bibitem[\citeproctext]{ref-acemoglu2024simple}
Acemoglu, Daron. 2024. {``The Simple Macroeconomics of AI.''} National
Bureau of Economic Research.
\url{https://economics.mit.edu/sites/default/files/2024-04/The\%20Simple\%20Macroeconomics\%20of\%20AI.pdf}.

\bibitem[\citeproctext]{ref-acemoglu2018race}
Acemoglu, Daron, and Pascual Restrepo. 2018. {``The Race Between Man and
Machine: Implications of Technology for Growth, Factor Shares, and
Employment.''} \emph{American Economic Review} 108 (6): 1488--1542.

\bibitem[\citeproctext]{ref-acemoglu2020unpacking}
---------. 2020. {``Unpacking Skill Bias: Automation and New Tasks.''}
w26681. National Bureau of Economic Research.

\bibitem[\citeproctext]{ref-acemoglu2022tasks}
---------. 2022. {``Tasks, Automation, and the Rise in US Wage
Inequality.''} \emph{Econometrica} 90 (5): 1973--2016.

\bibitem[\citeproctext]{ref-adachi2024robots}
Adachi, Daisuke, Daiji Kawaguchi, and Yukiko U Saito. 2024. {``Robots
and Employment: Evidence from Japan, 1978--2017.''} \emph{Journal of
Labor Economics} 42 (2): 591--634.

\bibitem[\citeproctext]{ref-aghion2024ai}
Aghion, Philippe, and Simon Bunel. 2024. {``AI and Growth: Where Do We
Stand.''}
\url{https://www.frbsf.org/wp-content/uploads/AI-and-Growth-Aghion-Bunel.pdf}.

\bibitem[\citeproctext]{ref-aghion2019artificial}
Aghion, Philippe, Benjamin F. Jones, and Charles I. Jones. 2019.
{``Artificial Intelligence and Economic Growth.''} In \emph{An Agenda},
edited by Ajay Agrawal, Joshua Gans, and Avi Goldfarb, 237--90. Chicago:
University of Chicago Press.
\url{https://doi.org/doi:10.7208/9780226613475-011}.

\bibitem[\citeproctext]{ref-agrawal2023turing}
Agrawal, Ajay K, Joshua S Gans, and Avi Goldfarb. 2023. {``The Turing
Transformation: Artificial Intelligence, Intelligence Augmentation, and
Skill Premiums.''} National Bureau of Economic Research.

\bibitem[\citeproctext]{ref-agrawal2023automation}
Agrawal, Ajay, Joshua S Gans, and Avi Goldfarb. 2023. {``Do We Want Less
Automation?''} \emph{Science} 381 (6654): 155--58.

\bibitem[\citeproctext]{ref-autor2024applying}
Autor, David. 2024. {``Applying AI to Rebuild Middle Class Jobs.''}
National Bureau of Economic Research.

\bibitem[\citeproctext]{ref-autor2024new}
Autor, David, Caroline Chin, Anna Salomons, and Bryan Seegmiller. 2024.
{``New Frontiers: The Origins and Content of New Work, 1940--2018.''}
\emph{The Quarterly Journal of Economics}, qjae008.

\bibitem[\citeproctext]{ref-autor2021persistence}
Autor, David, David Dorn, and Gordon H Hanson. 2021. {``On the
Persistence of the China Shock.''} National Bureau of Economic Research.

\bibitem[\citeproctext]{ref-autor2020fall}
Autor, David, David Dorn, Lawrence F Katz, Christina Patterson, and John
Van Reenen. 2020. {``{The Fall of the Labor Share and the Rise of
Superstar Firms*}.''} \emph{The Quarterly Journal of Economics} 135 (2):
645--709. \url{https://doi.org/10.1093/qje/qjaa004}.

\bibitem[\citeproctext]{ref-autor2003skill}
Autor, David, Frank Levy, and Richard J Murnane. 2003. {``The Skill
Content of Recent Technological Change: An Empirical Exploration.''}
\emph{The Quarterly Journal of Economics} 118 (4): 1279--1333.

\bibitem[\citeproctext]{ref-baqaee2019macro}
Baqaee, David Rezza, and Emmanuel Farhi. 2019. {``The Macroeconomic
Impact of Microeconomic Shocks: Beyond Hulten's Theorem.''}
\emph{Econometrica} 87 (4): 1155--1206.

\bibitem[\citeproctext]{ref-barnett2025consequences}
Barnett, Matthew. 2025. {``The Economic Consequences of Automating
Remote Work.''}
\url{https://epoch.ai/gradient-updates/consequences-of-automating-remote-work}.

\bibitem[\citeproctext]{ref-baumol1965performing}
Baumol, W. J., and W. G. Bowen. 1965. {``On the Performing Arts: The
Anatomy of Their Economic Problems.''} \emph{The American Economic
Review} 55 (1/2): 495--502. \url{http://www.jstor.org/stable/1816292}.

\bibitem[\citeproctext]{ref-benzell2022digital}
Benzell, Seth G, Erik Brynjolfsson, and Guillaume Saint-Jacques. 2022.
{``Digital Abundance Meets Scarce Architects: Implications for Wages,
Interest Rates, and Growth.''}

\bibitem[\citeproctext]{ref-beraja2022inefficient}
Beraja, Martin, and Nathan Zorzi. 2022. {``Inefficient Automation.''}
National Bureau of Economic Research.

\bibitem[\citeproctext]{ref-bessen2016computer}
Bessen, James E. 2016. {``How Computer Automation Affects Occupations:
Technology, Jobs, and Skills.''} \emph{Boston Univ. School of Law, Law
and Economics Research Paper}, no. 15-49.

\bibitem[\citeproctext]{ref-bessen2023automatic}
Bessen, James, Martin Goos, Anna Salomons, and Wiljan Van den Berge.
2023. {``Automatic Reaction-What Happens to Workers at Firms That
Automate?''} \emph{The Review of Economics and Statistics}, no. Feb. 6,
2023.

\bibitem[\citeproctext]{ref-boppart2020labor}
Boppart, Timo, and Per Krusell. 2020. {``Labor Supply in the Past,
Present, and Future: A Balanced-Growth Perspective.''} \emph{Journal of
Political Economy} 128 (1): 118--57.

\bibitem[\citeproctext]{ref-brynjolfsson2023turing}
Brynjolfsson, Erik. 2023. {``The Turing Trap: The Promise \& Peril of
Human-Like Artificial Intelligence.''} In \emph{Augmented Education in
the Global Age}, 103--16. Routledge.

\bibitem[\citeproctext]{ref-brynjolfsson2023generative}
Brynjolfsson, Erik, Danielle Li, and Lindsey R Raymond. 2023.
{``Generative AI at Work.''} \emph{Available at SSRN 4573321}.

\bibitem[\citeproctext]{ref-buera2015skill}
Buera, Francisco J, Joseph P Kaboski, and Richard Rogerson. 2015.
{``Skill-Biased Structural Change.''} \emph{American Economic Journal:
Macroeconomics} 7 (3): 95--150.

\bibitem[\citeproctext]{ref-chow2023transformative}
Chow, Trevor, Basil Halperin, and J Zachary Mazlish. 2023.
{``Transformative AI, Existential Risk, and Asset Pricing.''} Working
Paper.

\bibitem[\citeproctext]{ref-costinot2023robots}
Costinot, Arnaud, and Ivan Werning. 2023. {``Robots, Trade, and Luddism:
A Sufficient Statistic Approach to Optimal Technology Regulation.''}
\emph{The Review of Economic Studies} 90 (5): 2261--91.

\bibitem[\citeproctext]{ref-davidson2021could}
Davidson, Tom. 2021. {``Could Advanced AI Drive Explosive Economic
Growth.''} \emph{Open Philanthropy} 25.

\bibitem[\citeproctext]{ref-dixit1977monopolistic}
Dixit, Avinash K, and Joseph E Stiglitz. 1977. {``Monopolistic
Competition and Optimum Product Diversity.''} \emph{The American
Economic Review} 67 (3): 297--308.

\bibitem[\citeproctext]{ref-eisfeldt2023labor}
Eisfeldt, Andrea L, Gregor Schubert, Miao Ben Zhang, and Bledi Taska.
2023. {``The Labor Impact of Generative AI on Firm Values.''}
\emph{Available at SSRN 4436627}.

\bibitem[\citeproctext]{ref-erdil2023explosive}
Erdil, Ege, and Tamay Besiroglu. 2023. {``Explosive Growth from AI
Automation: A Review of the Arguments.''} \emph{arXiv Preprint
arXiv:2309.11690}.

\bibitem[\citeproctext]{ref-gechert2022measuring}
Gechert, Sebastian, Tomas Havranek, Zuzana Irsova, and Dominika
Kolcunova. 2022. {``Measuring Capital-Labor Substitution: The Importance
of Method Choices and Publication Bias.''} \emph{Review of Economic
Dynamics} 45: 55--82.

\bibitem[\citeproctext]{ref-giuntella2022workers}
Giuntella, Osea, Yi Lu, and Tianyi Wang. 2022. {``How Do Workers and
Households Adjust to Robots? Evidence from China.''} National Bureau of
Economic Research.

\bibitem[\citeproctext]{ref-hemous2022rise}
Hemous, David, and Morten Olsen. 2022. {``The Rise of the Machines:
Automation, Horizontal Innovation, and Income Inequality.''}
\emph{American Economic Journal: Macroeconomics} 14 (1): 179--223.

\bibitem[\citeproctext]{ref-hotte2023technology}
Hotte, Kerstin, Melline Somers, and Angelos Theodorakopoulos. 2023.
{``Technology and Jobs: A Systematic Literature Review.''}
\emph{Technological Forecasting and Social Change} 194: 122750.

\bibitem[\citeproctext]{ref-hulten1978growth}
Hulten, Charles R. 1978. {``Growth Accounting with Intermediate
Inputs.''} \emph{The Review of Economic Studies} 45 (3): 511--18.

\bibitem[\citeproctext]{ref-humlum2019robot}
Humlum, Anders. 2019. {``Robot Adoption and Labor Market Dynamics.''}
\emph{Princeton University}. \url{https://www.nber.org/papers/w33777}.

\bibitem[\citeproctext]{ref-ide2024artificialintelligenceknowledgeeconomy}
Ide, Enrique, and Eduard Talamas. 2024a. {``Artificial Intelligence in
the Knowledge Economy.''} \url{https://arxiv.org/abs/2312.05481}.

\bibitem[\citeproctext]{ref-ide2024turingvalleyaicapabilities}
---------. 2024b. {``The Turing Valley: How AI Capabilities Shape Labor
Income.''} \url{https://arxiv.org/abs/2408.16443}.

\bibitem[\citeproctext]{ref-ilo2015income}
ILO, IMF, OECD. 2015. {``Income Inequality and Labour Income Share in
G20 Countries: Trends, Impacts and Causes.''}

\bibitem[\citeproctext]{ref-jones2023ai}
Jones, Charles I. 2023. {``The AI Dilemma: Growth Versus Existential
Risk.''} National Bureau of Economic Research.

\bibitem[\citeproctext]{ref-karabarbounis2014global}
Karabarbounis, Loukas, and Brent Neiman. 2014. {``The Global Decline of
the Labor Share.''} \emph{The Quarterly Journal of Economics} 129 (1):
61--103.

\bibitem[\citeproctext]{ref-katz1992changes}
Katz, Lawrence F, and Kevin M Murphy. 1992. {``Changes in Relative
Wages, 1963--1987: Supply and Demand Factors.''} \emph{The Quarterly
Journal of Economics} 107 (1): 35--78.

\bibitem[\citeproctext]{ref-korinek2024scenarios}
Korinek, Anton, and Donghyun Suh. 2024. {``Scenarios for the Transition
to AGI.''} National Bureau of Economic Research.
\url{https://arxiv.org/abs/2403.12107}.

\bibitem[\citeproctext]{ref-krusell2000capital}
Krusell, Per, Lee E Ohanian, Jose-Victor Rios-Rull, and Giovanni L
Violante. 2000. {``Capital-Skill Complementarity and Inequality: A
Macroeconomic Analysis.''} \emph{Econometrica} 68 (5): 1029--53.

\bibitem[\citeproctext]{ref-lehr2022optimal}
Lehr, Nils Haakon, and Pascual Restrepo. 2022. {``Optimal Gradualism.''}
National Bureau of Economic Research.

\bibitem[\citeproctext]{ref-lu2021review}
Lu, Yingying, and Yixiao Zhou. 2021. {``A Review on the Economics of
Artificial Intelligence.''} \emph{Journal of Economic Surveys} 35 (4):
1045--72.

\bibitem[\citeproctext]{ref-ngai2007structural}
Ngai, L Rachel, and Christopher A Pissarides. 2007. {``Structural Change
in a Multisector Model of Growth.''} \emph{American Economic Review} 97
(1): 429--43.

\bibitem[\citeproctext]{ref-nordhaus2021singularity}
Nordhaus, William D. 2021. {``Are We Approaching an Economic
Singularity? Information Technology and the Future of Economic
Growth.''} \emph{American Economic Journal: Macroeconomics} 13 (1):
299--332.

\bibitem[\citeproctext]{ref-noy2023generative}
Noy, Shakked, and Whitney Zhang. 2023. {``Experimental Evidence on the
Productivity Effects of Generative AI.''} \emph{arXiv Preprint
arXiv:2304.02313}.

\bibitem[\citeproctext]{ref-oberfield2021micro}
Oberfield, Ezra, and Devesh Raval. 2021. {``Micro Data and Macro
Technology.''} \emph{Econometrica} 89 (2): 703--32.

\bibitem[\citeproctext]{ref-ricardo1821principles}
Ricardo, David. 1821. \emph{On the Principles of Political Economy}. J.
Murray London.

\bibitem[\citeproctext]{ref-sachs2012smart}
Sachs, Jeffrey D, and Laurence J Kotlikoff. 2012. {``Smart Machines and
Long-Term Misery.''} National Bureau of economic research.

\bibitem[\citeproctext]{ref-trammell2023economic}
Trammell, Philip, and Anton Korinek. 2023. {``Economic Growth Under
Transformative AI.''} National Bureau of Economic Research.

\bibitem[\citeproctext]{ref-zeira1998workers}
Zeira, Joseph. 1998. {``Workers, Machines, and Economic Growth.''}
\emph{The Quarterly Journal of Economics} 113 (4): 1091--1117.

\end{CSLReferences}




\end{document}
